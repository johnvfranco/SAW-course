\documentclass[12pt]{article}
\input{epsf}
\usepackage{color,amsmath, amsthm, amssymb, hyperref}
\textwidth=6.50in
\textheight=10in
\evensidemargin=-0in
\oddsidemargin=-0in
\topmargin=-.4in
\pagestyle{empty}
\font\bg=cmmib10 at 20pt
\font\sbg=cmmib10 at 18pt
\font\medb=cmssbx10 at 18pt
\font\norm=cmss10 at 13pt
\font\norml=cmssi10 at 13pt
\font\smm=cmr8 at 8pt
\font\tib=cmb10 at 26pt
\newcommand{\comp}{}
\newcommand{\fa}{\psi_1}
\newcommand{\fb}{\psi_2}
\newcommand{\vecalpha}{{\bg\symbol{11}}~}
\newcommand{\svecalpha}{{\sbg\symbol{11}}~}
\newcommand{\vecbeta}{{\bg\symbol{12}}~}
\newcommand{\vecgamma}{{\bg\symbol{13}}~}
\def\comp#1{\overline{#1}}
\def\clausebox(#1,#2)#3{\makebox(0,0){\framebox(#1,#2){${#3}$}}}
\def\Box{\large\fbox{$^{~}$}}
\def\Sbox{\tiny\fbox{$^{~}$}}
\def\henceJS{\mathord{{\Box}}}
\def\eventuallyJS{\mathord{\diamond}}
\def\nextJS{\mathord{\circ}}
\def\untilJS{\mathord{\hspace{1.5pt}{\cal U}\hspace{1.5pt}}}
\definecolor{LemonChiffon}{rgb}{1.,0.98,0.8}
\definecolor{title}{rgb}{0.0,0.0,0.73}
\definecolor{subtitle}{rgb}{0.0,0.0,0.5}
\definecolor{widgets}{rgb}{0.1,0.5,0.1}
\definecolor{dred}{rgb}{0.73,0.0,0.0}
\definecolor{dblue}{rgb}{0.0,0.0,0.5}
\definecolor{dgreen}{rgb}{0.0,0.3,0.0}
\definecolor{lgreen}{rgb}{0.0,0.6,0.0}
\definecolor{mag}{rgb}{0.7,0.0,0.7}
\definecolor{dmag}{rgb}{0.5,0.0,0.5}
\definecolor{wht}{rgb}{1,1,1}
\begin{document}

\begin{center}
%{\large\bf \textcolor{title}{\hfill 20-CS-[51$\mid$60]21\hfill Lab 6\hfill Spring 2023\hfill }}
\end{center}
\vspace*{-1mm}
\centerline{\Large\bf \textcolor{title}{Find Weak Keys of DES}}
\vspace*{6mm}
The DES specification in cryptol is provided in\\
{\tt http://gauss.ececs.uc.edu/Courses/c6021/labs/DES.cry} and\\
{\tt http://gauss.ececs.uc.edu/Courses/c6021/labs/Cipher.cry}\\
Add a property that will allow you to find weak keys for DES.\\
Find all of them and submit your modified {\tt DES.cry}.\\

\noindent
{\bf Definition:} {\em weak keys of DES}: keys that result in all per-round
keys being identical.\\

\noindent
    {\bf Help:} in {\tt DES.cry} there is a function called {\tt expandKey} that
    takes a key as argument and produces a sequence of 16 per-round keys.  All
    you have to do is create a function that checks whether all 16 numbers
    in the sequence are the same.  Then create a property that compares the
    output of that function to something such that the comparison is True iff
    all 16 per-round subkeys are the same.

\vspace*{5mm}
\noindent
\hspace*{2mm}{\small Submission intructions: \url{http://gauss.ececs.uc.edu/Courses/c6021/labs/submit.html}}

\end{document}
